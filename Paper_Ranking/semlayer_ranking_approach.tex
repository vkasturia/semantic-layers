
In this section, we formalize the problem of ranking documents returned by
structured (SPARQL) queries and define a probabilistic ranking model.
First we introduce the required notions and notations.

\subsection{Notions and Notations}

\subsubsection{Entities.}
In our problem, an {\em entity} is anything with a separate and meaningful existence
that also has an identity expressed through a reference in a knowledge base (e.g., a Wikipedia/DBpedia URI).
This does not only include persons, locations, organizations, etc., but also
events (e.g., {\em US 2016 presidential election})
and more abstract concepts such as {\em democracy} or {\em apartheid}.
Let $E$ be a finite set of entities, e.g., all Wikipedia entities,
where each entity $e \in E$ is associated with a unique URI in the reference knowledge base.

\comment{====
    \vspace{2mm} \noindent
    {\bf Documents and extracted entities}.
    Let $D$ be a set of documents, e.g., a set of news articles.
    For a document $d \in D$, let $publ(d)$ denote its publication date.
    Let also  $ents(d) \subseteq E$ be all entities mentioned in $d$.
    %(actually the URIs of these entities\footnote{From now on, when we say \q{entity} we mean \q{web entity URI}.}).
    Inversely, for an entity $e \in E$, let $docs(e) \subseteq D$
    be all documents that mention $e$, i.e., $docs(e) = \{d \in D ~|~ e \in ents(d)\}$.
    For an entity $e \in E$ and a document $d \in D$,
    let $count(e, d)$ be the number of occurrences of $e$ in $d$.
    %Now, let $ef(e, d) = \frac{count(e, d)}{\sum_{e' \in ents(d)}{count(e', d)}}$ be the
    %(normalized) frequency of $e$ in $d$.
    Finally, let $E_D$ be all entities mentioned in documents of $D$,
    i.e., $E_D = \cup_{d \in D}{ents(d)}$.

    \vspace{2mm} \noindent
    {\bf Time periods}.
    Let $\Delta$ be a fixed time period granularity (e.g., {\em day}).
    Based on a granularity $\Delta$,
    let $T = (t_0, t_1, \dots, t_n)$ be an ordered list of consecutive time points
    covering the publication dates of all documents in $D$.
    Note that $t_i - t_{i-1} = \Delta$, for each $i \geq 1$.
    Now, let $\delta_i = [t_i, t_{i+1})$ be the time period of size $\Delta$ between two consecutive time points.
    We consider that a document $d$ is published in the period $\delta_i$, if $t_i \leq publ(d) < t_{i+1}$.
    For a document $d \in D$, let $p_d$ be the time period in which $d$ was published.
    Now, let $P_D$ denote the set of distinct time periods of all documents in $D$,
    i.e., $P_D = \cup_{d \in D}{\{p_d\}}$.
    For a time period $p \in P_D$, let $docs(p) \subseteq D$ be the set of all
    documents published within $p$, i.e., $docs(p) = \{d \in D ~|~ p_d = p\}$,
    and $ents(p) \subseteq E_D$ be the set of
    entities discussed in documents of $D$ published within $p$,
    i.e., $ents(p) = \cup_{d \in docs(p)}{ents(d)}$.
====}

\subsubsection{Documents and extracted entities.}
Let $D$ be a set of documents (e.g., a set of news articles)
published within a set of time periods $T_D$ of fixed granularity $\Delta$ (e.g., day).
For a document $d \in D$, let $t_d \in T_D$ be the time period of granularity $\Delta$ in which $d$ was published,
while for a time period $t \in T_D$, let $docs(t) \subseteq D$ be the set of all
documents published within $t$, i.e., $docs(t) = \{d \in D ~|~ t_d = t\}$.

Let now  $ents(d) \subseteq E$ be all entities mentioned in $d$.
Inversely, for an entity $e \in E$, let $docs(e) \subseteq D$
be all documents that mention $e$, i.e., $docs(e) = \{d \in D ~|~ e \in ents(d)\}$.


%and $ents(p) \subseteq E_D$ be the set of
%entities discussed in documents of $D$ published within $p$
%(i.e., $ents(p) = \cup_{d \in docs(p)}{ents(d)}$).


\subsection{Problem Definition}
Given a corpus of documents $D$,
a set of entities $E_D$ mentioned in documents of $D$,
and a SPARQL query $Q$ requesting documents from $D$
published within a {\em time period} $T_Q \subseteq T_D$ and
related to one or more query entities $E_Q \subseteq E_D$
with logical {\tt AND} (mentioning all the query entities) or {\tt OR} (mentioning at least one of the query entities) semantics,
the problem is how to rank the documents $D_Q \subseteq D$ that (equally) match $Q$.

Figure \ref{fig:modelingExampleQ1} shows an example SPARQL query
requesting documents published in 1990
discussing about the entities {\em Nelson Mandela} %(\url{http://dbpedia.org/resource/Nelson_Mandela})
and {\em Frederik Willem de Klerk} %(\url{http://dbpedia.org/resource/F._W._de_Klerk}).
({\tt AND} semantics),
while the query in Figure \ref{fig:modelingExampleQ2} requests
articles of 1990 discussing about {\em state presidents of South Africa}
({\tt OR} semantics).
Our objective is to rank the results returned by such SPARQL queries.


\begin{figure}[th]
\vspace{-4mm}
\centering \scriptsize
\begin{Verbatim}[frame=lines,numbers=left,numbersep=1pt]
SELECT DISTINCT ?article WHERE {
  ?article dc:date ?date FILTER(?date >= "1990-01-01"^^xsd:date &&
                                ?date <= "1990-12-31"^^xsd:date) .
  ?article oae:mentions ?entity1, ?entity2 .
  ?entity1 oae:hasMatchedURI  <http://dbpedia.org/resource/Nelson_Mandela> .
  ?entity2 oae:hasMatchedURI  <http://dbpedia.org/resource/F._W._de_Klerk> }
\end{Verbatim}
\vspace{-5.5mm}
\caption{SPARQL query for retrieving articles of 1990 discussing
about {\em Nelson Mandela} and {\em Frederik Willem de Klerk} ({\tt AND} semantics).}
\label{fig:modelingExampleQ1}

%\vspace{-4mm}
\centering \scriptsize
\begin{Verbatim}[frame=lines,numbers=left,numbersep=1pt]
SELECT DISTINCT ?article WHERE {
  SERVICE <http://dbpedia.org/sparql> {
    ?p dc:subject <http://dbpedia.org/resource/Category:State_Presidents_of_South_Africa> }
  ?article dc:date ?date FILTER(?date >= "1990-01-01"^^xsd:date &&
                                ?date <= "1990-12-31"^^xsd:date)
  ?article oae:mentions ?entity .
  ?entity oae:hasMatchedURI  ?p }
\end{Verbatim}
\vspace{-5.5mm}
\caption{SPARQL query for retrieving articles of 1990 discussing
about {\em state presidents of South Africa} ({\tt OR} semantics).}
\label{fig:modelingExampleQ2}
\vspace{-4mm}
\end{figure}



\subsection{Probabilistic Modeling}

We model and estimate the probability to select a document $d \in D_Q$
given the query entities $E_Q$, the time period of interest $T_Q$,
and other entities $E_{D_Q}$ mentioned in the retrieved documents.
Specifically, we model the following aspects:
\begin{itemize}
\item the {\em relativeness} of a document with respect to the query entities
\item the {\em timeliness} of a document with respect to its publication date
\item the {\em relatedness} of a document with respect to its reference to other entities related to the query entities
\end{itemize}
% iv) the {\em temporal closeness} of a document with respect to other documents.}
The idea is to promote documents that
mention the query entities many times in their contents,
have been published in important (for the query entities) time periods, and
mention many other entities that co-occur frequently with the query entities in important time periods.
For example, in case we want to rank articles of 1990
discussing about {\em Nelson Mandela},
we want to favor articles that
i) mention {\em Nelson Mandela} multiple times in their text,
ii) have been published in important (for {\em Nelson Mandela}) time periods
(e.g., February of 1990 since during that period he was released from prison), and
iii) mention other entities that seem to be important for Nelson Mandela
during important time periods
(e.g., {\em Frederik Willem de Klerk} who was
South Africa's State President in 1990).

\subsection*{Relativeness}

We consider that if the query entities are mentioned multiple times
within a document, the document should receive a high score
(since the document's topic may be about these entities).
The term frequency (in our case entity frequency) is a classic numerical statistic
that is intended to reflect how important
a word is to a document in a collection or corpus \cite{leskovec2014mining}.

We first define a {\em relativeness} score of a document $d \in D_Q$
based on the {\em frequency} of the query entities in $d$.
First, let $count(e, d)$ be the number of occurrences of $e$ in document $d$.
For the case of {\tt AND} semantics (\q{$\wedge$}),
the score is defined as follows:
\begin{equation}
\small
score^{f}_{\wedge}(d, E_Q) = \frac{\sum_{e \in E_Q}{count(e, d)}}{\sum_{e' \in ents(d)}{count(e', d)}}
\end{equation}
Notice that the score of a document will be $1$ if it contains the query entities and no other entity.
For the case of {\tt OR} semantics (\q{$\vee$}),
we can also consider the number of query entities mentioned in the document
(since a document does not probably contain all the query entities as in the case of {\tt AND} semantics).
In that case, the {\em relativeness} score can be defined as follows:
\begin{equation}
\small
score^{f}_{\vee}(d, E_Q) = \frac{\sum_{e \in E_Q}{count(e, d)}}{\sum_{e' \in ents(d)}{count(e', d)}} \cdot \frac{|ents(d) \cap E_Q|}{|E_Q|}
\end{equation}
where $\frac{|ents(d) \cap E_Q|}{|E_Q|}$ is the percentage of query entities discussed in the document.
The score of a document will be 1 if it contains all the query entities and no other entity.
This formula favors documents mentioning multiple times many of the query entities.

Now, the probability of a retrieved document $d \in D_Q$ given only the query entities can be defined as:
\begin{equation}
\small
P(d|E_Q) = \frac{score^{f}(d, E_Q)}{\sum_{d' \in D_Q}{score^{f}(d', E_Q)}}
\end{equation}
%Notice that the above formula is a probability distribution of all returned documents.


\subsection*{Timeliness}

We consider that a time period $t \in T_Q$ is important
for the entities in $E_Q$, if there is a relatively big number of documents in $D_Q$
discussing about these entities during $t$.
For example, a big number of articles about {\em Nelson Mandela}
was published the period 11-13 of February 1990 because
in February 11 {\em Nelson Mandela} was released from prison.
Thus, articles published during that period should be promoted since
they are probably related to this important event of {\em Nelson Mandela}'s life.

For the case of {\tt AND} semantics,
we define the following importance score of a {\em time period} $t \in T_Q$:
\begin{equation}
score^{t}_\wedge(t) = \frac{|docs(t) \cap D_Q|}{|D_Q|}
\end{equation}
This scoring formula favors time periods in which there is a big number
of documents discussing about the query entities.

For the case of {\tt OR} semantics,
in a time period $t$ there may be a
big number of documents discussing only for one of the query entities,
while in another time period $t'$ there may be a smaller number
of documents discussing though for many of the query entities.
For also taking into account the number of query entities discussed in documents
of a specific time period, we consider the following formula:
\begin{equation}
\small
score^{t}_\vee(t) = \frac{|docs(t) \cap D_Q|}{|D_Q|} \cdot N(E_Q,t)
\end{equation}
where, $N(E_Q,t)$ is the average percentage of query entities discussed in articles of $t$, i.e.:
\begin{equation}
\small
N(E_Q,t) =  \frac{\sum_{d \in docs(t) \cap D_Q}{\frac{|ents(d) \cap E_Q|}{|E_Q|}}}{|docs(t) \cap D_Q|}
\end{equation}

Now, the probability of a retrieved document $d \in D_Q$ given only its publication date $t_d$ can be defined as:

\begin{equation}
\small
P(d | t_d) = \frac{score^{t}(t_d)}{\sum_{d' \in D_Q}{score^{t}(t_{d'})}}
\end{equation}



\subsection*{Relatedness}

Entities that are co-mentioned frequently with the query entities
in important time periods are probably important.
For example, {\em Apartheid} was an important concept
related to {\em Nelson Mandela} %and {\em Frederik Willem de Klerk}
during 1990,
thus articles discussing for both {\em Apartheid} and {\em Nelson Mandela} %, and {\em Frederik Willem de Klerk}
should be promoted.
However, there may be also some general entities (e.g., {\em South Africa} in our example)
that co-occur with the query entities in almost all documents (independently of the time period).
Thus, we should also avoid over-emphasizing documents mentioning such \q{common} entities.
%(in a similar way to the computation of {\em tf-idf} in IR).

For the case of {\tt AND} semantics,
we consider the following {\em relatedness} score
for an entity $e \in E_D \setminus E_Q$:

\begin{equation}
\begin{split}
\small
score^{r}_\wedge(e) = & ~ idf_\wedge(e) \cdot \sum_{t \in T_Q}{(score^{t}_\wedge(t) \cdot \frac{|docs(t) \cap D_Q \cap docs(e)|}{|docs(t) \cap D_Q|})}\\
= & ~ idf_\wedge(e) \cdot \sum_{t \in T_Q}{\frac{|docs(t) \cap D_Q \cap docs(e)|}{|D_Q|}}
\end{split}
\end{equation}
where $idf_\wedge(e)$ is the inverse document frequency
of entity $e$ in the set of documents discussing about the query entities in the entire corpus,
which can be defined as follows:
\begin{equation}
\label{frml:idf_and}
\small
idf_\wedge(e) = 1 - \frac{|docs(e) \cap (\cap_{e' \in E_Q}{docs(e')})|}{|\cap_{e' \in E_Q}{docs(e')}|}
\end{equation}
The formula considers the percentage of
documents in which the entity
co-occurs with the query entities in important time periods.

For the case of {\tt OR} semantics,
the above formula does not consider the
number of different query entities discussed in documents together with the entity $e$.
To also handle this aspect,
we consider the following {\em relatedness} score for the case of {\tt OR} semantics:
\begin{equation}
\begin{split}
\small
score^{r}_\vee(e) =  & ~ idf_\vee(e) \cdot  N(E_Q, e) \cdot \sum_{t \in T_Q}{(score^{t}_\vee(t) \cdot \frac{|docs(t) \cap D_Q \cap docs(e)|}{|docs(t) \cap D_Q|})} \\
= & ~ idf_\vee(e) \cdot N(E_Q, e) \cdot  \sum_{t \in T_Q}{(N(E_Q, t)\cdot \frac{|docs(t) \cap D_Q \cap docs(e)|}{|D_Q|})}
\end{split}
\end{equation}
where $N(E_Q, e)$ is
the average percentage of query entities discussed in articles together with entity $e$, i.e.:
\begin{equation}
\small
N(E_Q, e) =  \frac{\sum_{d \in docs(e) \cap D_Q}{\frac{|ents(d) \cap E_Q|}{|E_Q|}}}{|docs(e) \cap D_Q|}
\end{equation}

Now the inverse document frequency $idf_\vee(e)$
includes documents mentioning at least one of the query entities, i.e.:
\begin{equation}
\label{frml:idf_or}
\small
idf_\vee(e) = 1 - \frac{|docs(e) \cap (\cup_{e' \in E_Q}{docs(e')})|}{|\cup_{e' \in E_Q}{docs(e')}|}
\end{equation}


This formula favors related entities that
i)  co-occur frequently with many of the query entities,
ii) are discussed in documents published in important (for the query entities) time periods.

Now, the probability of a document $d \in D_Q$
given only other entities mentioned in the retrieved documents ($E_{D_Q}$) can be defined as:
\begin{equation}
P(d | E_{D_Q}) = \frac{\sum_{e \in ents(d) \setminus E_Q}{score^{r}(e)}}
                  {\sum_{d' \in D_Q}{\sum_{e' \in ents(d') \setminus E_Q}{score^{r}(e')}}}
\end{equation}

\comment{====
I PUT THE FOLLOWING IN COMMENTS SINCE
WE HAVE NOT EVALUATED IT.
\new{
\subsection*{Temporal Closeness}
The archived documents form important information sources
for mainly historians, journalists and sociologists.
To a historian documenting an event in history
or tracing the occurrence of the event,
if she finds a document mentioning about the event,
to get more detail she would more likely try to find documents published
shortly before or after it.
As an example, take a historian researching Nelson Mandela's release from prison and
submitting a query to find articles of February 1990 mentioning him.
She stumbles first upon an article of 11th February as that was the day Nelson Mandela
was released.
To get a complete overview of the event, she would more likely search for documents
published in the close past or future to it, i.e., documents mentioning protests
and announcements leading to his release
and celebrations or political implications afterwards.
Hence, documents published in closer
date proximity to an important document gain more importance than other documents
in this case.
We try to capture this by defining temporal closeness of a document d' relative to a document d
for the case of both {\tt AND} and {\tt OR} Semantics as follows:
\begin{equation}
\begin{split}
\small
ScoreD^c(d) = 1 - \frac{(|publ(d)-publ(d')|+1)}{\max_{d'' \in D_Q, d'' \neq d}(|publ(d)-publ(d'')|+1)}
\end{split}
\end{equation}
where $|publ(d)-publ(d')|$ is defined as the absolute difference (in {\em days})
of the publication dates of the documents $d$ and $d'$.
}
======}

\subsection*{Joining the Models}

We can now combine the different models in a single probability score:
\begin{equation}
\small
P(d | E_Q, t_d, E_{D_Q}) = \frac{P(d | E_Q)P(d | T_Q)P(d | E_{D_Q})}{\sum_{d' \in D_Q}{P(d' | E_Q)P(d' | T_Q)P(d' | E_{D_Q})}}
\end{equation}
where the denominator can be ignored as it does not influence the ranking.

\comment{====
    By considering {\em relativeness} and {\em timeliness},
    the score of a document $d \in D_Q$ can be defined as:
    \begin{equation}
    \small
    ScoreD^{f,t}(d) =  ScoreD^{f}(d) \cdot ScoreD^{t}(d)
    \end{equation}

    \noindent
    By considering {\em relativeness} and {\em relatedness},
    the score of a document $d \in D_Q$ can be defined as:
    \begin{equation}
    \small
    ScoreD^{f,r}(d) =  ScoreD^{f}(d) \cdot ScoreD^{r}(d)
    \end{equation}


    \noindent
    By considering all aspects,
    the score of a document $d \in D_Q$ can be defined as:
    \begin{equation}
    \small
    ScoreD^{f,t,r}(d) =  ScoreD^{f}(d) \cdot  ScoreD^{t}(d) + \beta ~ ScoreD^{r}(d)
    \end{equation}
    where $\beta$ is a decay factor for limiting the
    effect of relatedness.
====}

\comment{====
    {\em Ranking of query entities}.
    When we consider {\tt\em OR} semantics, a returned document $d \in D_Q$
    contains at least one of the query entities,
    while an entity $e \in E_Q$ can be discussed in zero, one or more documents.
    We can identify the important query entities in the whole time period $P_Q$
    by simply considering the percentage of documents in $D_Q$ discussing about each of the query entities,
    i.e: for an entity $e \in E_Q$, the importance score $I(e)$ can be defined as:
    $I(e) = \frac{|docs(e) \cap D_Q|}{|D_Q|}$.
    Likewise, we can compute the importance of an entity $e \in E_Q$ in a specific time period
    $p \in P_Q$ as:
    $I(e, p) = \frac{|docs(e) \cap docs(p)|}{|D_Q|}$.
====}
